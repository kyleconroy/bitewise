\documentclass[11pt,twocolumn]{article}

\usepackage{setspace}
\usepackage[left=1in,top=1in,right=1in,bottom=1in,nohead]{geometry}

\author{Kyle Conroy and Brandon Liu\\
http://www.github.com/derferman/bitewise}
\title{Bitewise: Ethnic Density and Restaurants Ratings}

\begin{document}
\maketitle

\section{Introduction}

Like many great ideas, our research topic came to us over lunch. Many San Francisco residents spend their lunch hour waiting in line in front of a food truck on the city streets. With the rise in popularity of these food trucks, we are interested in their history and success. Sadly, after looking into Data SF, the official data clearinghouse for the city and country of San Francisco, food truck data is in short support, with nothing beyond the location of such trucks.

Our investigation of food trucks sparked and inquiry into ethnic restaurants, particularly Mexican restaurants. We are curious about the relationship between the location and Yelp rating of a given burrito place. For example, are burrito places rated higher in the Mission than other neighborhoods? If so, than ethnic neighborhoods may tend to attract customers interested in finding quality cuisine of that ethnicity.

\section{Data ETL}

To answer this question, we use four different data sets. First, we use census data from the Federal Government to calculate demographic information about San Francisco and its surrounding areas. The United States census splits neighborhoods up into smaller plots of land named tracts. A tract is the right size for our project; in major cities, a tract spans just a few blocks. The City of San Francisco provides additional per-tract ethnicity information, allowing us to build an accurate ethnic representation of the entire city with decent resolution. This set is stored as a CSV with the tract, the ethnicity data, and the coordinate for the tract.

Combining these two datasets involved strange data formats not popular in today's world of web APIs. Census data comes in two flavors: plain-text ASCII format or a binary format called a shape file. We found that the ASCII format was too cumbersome to use, so we went with the binary format. The open-source Python Shapefile Library made parsing the census data quick and simple. Each entry in the shape file contains data about the entry and the latitude and longitude coordinates which define its shape. These points are used to construct a polygon representing the tract. The data associated with a shape entry contains useful information such as tract number and county number. We use this data to filter out those tracts which aren't in the San Francisco area.

Our other data sets come from Yelp. Since we're investigating business ratings, Yelp is the best place to go. Yelp provides a REST API for their data with the only requirement signing up for a developer key. Using the API, we collected business listings for six different categories of restaurant. These categories include Mexican, Chinese, and vegetarian. Our final restaurant data set includes over 1,000 restaurants in the San Francisco area. Almost half of those restaurants fall into the Mexican category.

Yelp returned the data back in JSON, making this data much easier to process than the census data. However, Yelp limits business results to twenty per request, so we had to combine multiple responses into one large list for each restaurant type. Yelp provided more information per business then we could analyze. Their location information proved to be especially useful, as it also includes a list of neighborhoods near each business.

Yelp returned the data in great shape, but there were a few missing pieces of information we had to collect ourselves. The Yelp API provides an image link for the rating of a business instead of an float value. Our data cleaning process involved converting these links into integers for further data processing. Many of the questions we wanted to ask revolved around the cost of a meal at each restaurant. However, Yelp doesn't included this data in their API results. To get the data, we scraped each individual business page on Yelp, parsed the HTML, performed a simple search over the page for the price, and updated our saved API results. Sadly, not all businesses have a price rating, so our data set remains incomplete in this area.

Our hypothesis requires placing businesses in their correct census tract. Geospactial queries can be time-consuming when using brute-force algorithms. To query over our census data, we use the Shapely python wrapper for GEOS, a popular open-source geometry engine. With these two tools, we quickly and efficiently found the census tract for each business. With this data, we can compute a score for ethnicity for each business. The score will be a function of nearby census tracts and ethnic makeup.

\section{Experiment 1: Comparison of Means}

The very first task perfomed to investigate our hypothesis was on three major ethnic neighborhoods in San Francisco: the Mission District, Chinatown, and Japantown. We  found the mean rating of all Chinese/Japanese/Mexican restaurants inside and outside their respective district, and compared using a Student's T-Test. 


\begin{center}
  \begin{tabular}{| l | c | c | c | r | }
    \hline
    Neighborhood & In & Out & n & p \\ \hline
    M/Mission & 5.81 & 4.98 & 303 & 1.8e-7 \\ \hline \hline
    C/Chinatown & 4.67 & 4.23 & 416 & 0.062 \\ \hline
    J/Japantown & 4.67 & 4.84 & 260 & 0.3539 \\
    \hline
  \end{tabular}
\end{center}

Means are reported in converted Yelp rating, where 1 star is mapped to 0 and 5 stars is mapped to 8. There are no half-star or 0 star averages in the Yelp system. The result of our T-tests was that: Mexican food is significantly better in its respective district; Chinese food and Japanese food did not have significant differences. We decided to probe on beyond a binary predictor of rating.

\section{Experiment 2: Numerical Predictor}

Neighborhoods are geographical boundaries reflecting historical geographic makeup. We thought that a better signal than the neighborhood of a restaurant was the ethnic makeup of the restaurant's surroundings. Thus, we used the percent of the respective ethnicity (Asian, Black, White, Hispanic) in the 2010 Census as a predictor for the Yelp rating. The reason we used percent instead of quantity was that census tracts are approximated to 4000 people; thus, by using quantity, one would be approximating percentage. We fitted a linear model to each of Chinese, Japanese, Korean, Mexican, Thai and Vietnamese food. 

\begin{center}
  \begin{tabular}{| l | c | c | c | c | c | r | }
    \hline
    Food & Itrcpt & coef(p) & dof & p & RSE & Rsq \\ \hline
    M & 4.64 & 1.96 & 2e-16 & 265 & 1.122 & 0.126 \\ \hline

    \hline
  \end{tabular}
\end{center}



Further questions:
Do the categories Vegetarian, Vegan vary with respect to the census data?

Does a different measure other than percentage act as a better predictor of the rating? For example, since tracts are around 4000 population, using the raw number of the self-ethnicity rather than a percentage may say something slightly different (how the rating depends on the pop density as well)

Does adding the price or number of reviews to the model improve the fit? We have to find a way to incorporate the price into the dataset; this is not part of the Yelp API.

\section{Casual Inference}

For our first study, we chose to only look at Mexican restaurants, plot them in a census tract, and then graph their yelp rating against the percentage of that ethnicity. We fitted a linear model to this and found that living in an ethnic area increased ratings by about half a star. There is a notable dearth of restaurants in the high-percent, low-rating quadrant. There are two noticeable clusters of restaurants; low-percentage and average rating, and high-percentage and high rating.

We decided to drill down into the data more. We created a histogram of all the ratings and saw that these were roughly normally distributed. Then, we looked at a histogram of percentages and were surprised to see that for the Hispanic population, these look bimodal. There is a peak around 20\%, and also a peak around 70\%. This may be an indicator of an ethnic neighborhood. Next, we intend to do a paired study using number of reviews as a covariate, and pairing between “high” percent and low percent to perhaps find a stronger correlation.

Continuing with our analysis, we found similar results for Chinese food and Asian populations, even stronger than with Mexican food. However, the distribution of Asians was unimodal. Additionally, Vietnamese and Japanese food seemed to have a negative correlation between percentage and rating (as far as Asian populations go in general) it would be really helpful if the Census broke it down into more detailed ethnicity.

\section{Results}
MOAR here


\section{Data Product}
\end{document}
